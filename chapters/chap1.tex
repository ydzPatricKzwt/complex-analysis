% 建议使用 XeLaTeX 编译,以支持 ctexart 宏包
\part{复数与复变函数}
\section{复数的定义及其运算}

\begin{definition}[复数]
我们将复数定义为一对有序的实数 $(a,b)$ 。复数域记为 $\mathbb{C}$:
$$\mathbb{C}=\{(a,b):a\in\mathbb{R},b\in\mathbb{R}\}$$
并定义加法 $(a,b)+(c,d)=(a+c,b+d)$ 与乘法 $(a,b)(c,d)=(ac-bd,ad+bc)$ 。
\end{definition}

\begin{theorem}
复数域不是有序域 。
\end{theorem}

\begin{proof}
若 $\mathbb{C}$ 是有序域,则对于 $i \neq 0$,必有 $i>0$ 或 $i<0$ 。
\begin{itemize}
    \item 若 $i>0$,由有序域性质可得 $i^2 > 0 \cdot i \implies -1 > 0$,进而推导出 $0 > 1$,矛盾 。
    \item 若 $i<0$,同理可引出矛盾 。
\end{itemize}
\end{proof}

\section{复数的表示}

\begin{definition}[通常表示]
    设$i=\left( 0 ,1 \right) $, 那么:$\forall \left( a,b \right) \in \mathbb{C}, \, \left( a ,b \right) = a +b i$。记作$z= a + bi  $.
\end{definition}

\begin{definition}[几何表示]
设 $z=a+bi \neq 0$,那么:
 $z=r(\cos \theta + i \sin \theta)$ ,其中$\left( r, \theta \right) $ 是$\left( a, b \right) $ 的极坐标。
\end{definition}
几何表示相关定义\sn{记$Arg \, z = \theta $,且 $\arg z = Arg\,z$,如果 $Arg \in [\pi , \pi]$。}。

\begin{prop}[乘除法的意义]设$z_1, \, z_2 \in \bbC$,那么:
\begin{enumerate}
    \item $|z_1z_2| = |z_1||z_2|, \operatorname{Arg}(z_1z_2) = \operatorname{Arg}z_1 + \operatorname{Arg}z_2.$
    \item $ \left|\frac{z_1}{z_2}\right| = \frac{|z_1|}{|z_2|},  \operatorname{Arg}\left(\frac{z_1}{z_2}\right) = \operatorname{Arg}z_1 - \operatorname{Arg}z_2.$
\end{enumerate}
\end{prop}

\begin{exercise}
    在图\ref{fig:_1_png} 的三角形中, \(AB = AC\), \(PQ = RS\). \(M\) 和 \(N\) 分别是 \(PR\) 和 \(QS\) 的中点. 证明: \(MN \perp BC\).
\end{exercise}
\begin{marginfigure}
    \centering
    \includegraphics[width=0.8\textwidth]{figures/1.png}
    \caption{}
    \label{fig:_1_png}
\end{marginfigure}
\begin{exercise}
    证明: 平面上四点 \(z_1, z_2, z_3, z_4\) 共圆的充要条件为
\[
\operatorname{Im}\left( \frac{z_1 - z_3}{z_1 - z_4} \bigg/ \frac{z_2 - z_3}{z_2 - z_4} \right) = 0.
\]
\end{exercise}


\subsection{共轭复数}
\begin{definition}[共轭复数]
    设 $z = a + bi \in \bbC$ ,那么: 称 $\overline{z}$ 是$z$ 的共轭复数,如果 $\overline{z} = a - bi $。
\end{definition}

    \begin{prop}[共轭复数的性质]    设$z, w \in \bbC $,那么:
    \begin{enumerate}
    \item $\operatorname{Re} z = \frac{1}{2}(z + \overline{z}),\quad \operatorname{Im} z = \frac{1}{2\mathrm{i}}(z - \overline{z})$
    \item $z \overline{z} = \, \mid z \mid ^2 $
   \item \(\overline{z + w} = \overline{z} + \overline{w},\quad \overline{zw} = \overline{z}\overline{w}\)
  \item \(|zw| = |z||w|,\quad \left|\frac{z}{w}\right| = \frac{|z|}{|w|}\)
  \item \(|z| = |\overline{z}|\)
    \end{enumerate}
    \end{prop}

    \begin{prop}[不等式]设$z, w \in \bbC$,那么:
    \begin{enumerate}
        \item \( |\operatorname{Re} z| \leq |z| \),\( |\operatorname{Im} z| \leq |z| \);
\item \( |z + w| \leq |z| + |w| \),等号成立当且仅当存在某个 \( t \geq 0 \),使得 \( z = t w \);
\item \( |z - w| \geq \bigl||z| - |w|\bigr| \)。
    \end{enumerate}
    \end{prop}

    \subsection{de Mrivre公式与单位根}%
    \label{sec:de Mrivre}
    
    \begin{theorem}[de Moivre公式]
       $\left( \cos \theta + i \sin \theta \right) ^n = \cos n \theta +i \sin n \theta $。
    \end{theorem}
\begin{marginfigure}
    \centering
    \includegraphics[width=0.8\textwidth]{figures/2.jpg}
    \caption{单位根}
    \label{fig:_2_jpg}
\end{marginfigure}
    \begin{corollary}[n次方根]
        设$\omega= r\left( \cos \theta + i \sin \theta  \right) , z^n = \omega$,那么:\\
        $$z = \sqrt[n]{|w|}\left( \cos \frac{\theta + 2k\pi}{n} + \mathrm{i}\sin \frac{\theta + 2k\pi}{n} \right),\quad k = 0,1,\dots,n-1.$$
    \end{corollary}
 若记 \(\omega = \cos\frac{2\pi}{n} + \mathrm{i}\sin\frac{2\pi}{n}\), 则 \(\sqrt[n]{1}\) 的 \(n\) 个值为
\[
1, \omega, \omega^2, \dots, \omega^{n-1},
\]
称为 \(n\) 个单位根. 如果用 \(\sqrt[n]{w}\) 记 \(w\) 的任一 \(n\) 次根, 那么 \(w\) 的 \(n\) 个 \(n\) 次根又可表示为
\[
\sqrt[n]{w}, \sqrt[n]{w}\omega, \dots, \sqrt[n]{w}\omega^{n-1}.
\]
   
\section{复数的球面表示}

\begin{definition}[扩充平面]
    设$\bbC$为复平面,那么称$\bbC_{\infty}$ 为扩充平面/闭平面,如果$\bbC_{\infty} = \bbC \cup {\infty}$, 其中
$
z \pm \infty = \infty,\quad z \cdot \infty = \infty \ (z \neq 0), \,
\frac{z}{\infty} = 0,\quad \frac{z}{0} = \infty \ (z \neq 0)
$
,$0 \cdot \infty$ 和 $ \infty \pm \infty$ 都不规定其意义.
\end{definition}
此即复数域的一点紧化\sn{可以回忆一下一点紧化的概念和性质: 
\begin{definition}[拓扑空间的一点紧化]忘了\end{definition}   
\begin{prop}任何拓扑空间都是一点紧化的开子空间。 \end{prop}
}
对于复数域的一点紧化$\bbC_{\infty}$,常常仍然记为$\bbC$.
\begin{definition}[复数的球面表示]
    设单位球面 \( S = \{ (x_1, x_2, x_3) \in \mathbb{R}^3 : x_1^2 + x_2^2 + x_3^2 = 1 \} \),复数$z\in \bbC$,那么\[z_i = \frac{x_1 + \mathrm{i}x_2}{1 - x_3} \]
\end{definition}
\begin{proof}
    把 \(\mathbb{C}\) 等同于平面:
\[
\mathbb{C} = \{(x_1, x_2, 0): x_1, x_2 \in \mathbb{R}\}.
\]
固定 \(S\) 的北极 \(N\),即 \(N = (0,0,1)\)。对于 \(\mathbb{C}\) 上的任意点 \(z\),联结 \(N\) 和 \(z\) 的直线必和 \(S\) 交于一点 \(P\)(图 1.8)。若 \(|z| > 1\),则 \(P\) 在北半球上;若 \(|z| < 1\),则 \(P\) 在南半球上;若 \(|z| = 1\),则 \(P\) 就是 \(z\)。容易看出,当 \(z\) 趋向 \(\infty\) 时,球面上对应的点 \(P\) 趋向于北极 \(N\)。自然地,我们就把 \(\mathbb{C}_\infty\) 中的 \(\infty\) 对应于北极 \(N\)。这样一来,\(\mathbb{C}_\infty\) 中的所有点(包括无穷远点在内)都被移植到球面上去了,而在球面上,\(N\) 和其他的点是一视同仁的。
这样,从 \(z\) 便可算出它在球面上对应点的坐标。反过来,从球面上的点 \((x_1, x_2, x_3)\) 也可算出它在平面上的对应点 \(z\)。事实上,从上面的表达式得
\[
\begin{cases}
x_1 + \mathrm{i}x_2 = \frac{2z}{1 + |z|^2}, \\
1 - x_3 = \frac{2}{1 + |z|^2}.
\end{cases}
\]
由此即得
\[
z = \frac{x_1 + \mathrm{i}x_2}{1 - x_3}.
\]
\end{proof}

\begin{marginfigure}
    \centering
    \includegraphics[width=0.8\textwidth]{figures/3.png}
    \caption{}
    \label{fig:_3_png}
\end{marginfigure}

\section{复数列的极限}%
\label{sec:complex_limit}

\begin{definition}[圆盘]
    设$a\in \bbC, \, r > 0$,那么,称$B\left( a, r \right) $ 为以$a$ 为中心,$r$ 为半径的圆盘/a的邻域,如果 $B\left( a, r \right) = 
\begin{cases}
    \{z \in \bbC  \mid  \left| z-a \right| < r \}, &  a \neq \infty \\
    \{ z \in \bbC  \mid  \left| z \right| > r\}, &  a = \infty.
\end{cases}
$
\end{definition}

\begin{definition}[复数列的极限]
   称\(\lim_{n \to \infty} z_n = z_0\),如果 \( \forall  \varepsilon > 0\),当 \(n\) 充分大时,\(z_n \in B(z_0, \varepsilon)\).
\end{definition}
等价的说法是: \(\mathbb{C}\) 中的复数列 \(\{z_n\}\) 收敛到 \(\mathbb{C}\) 中的点 \(z_0\),是指对于任给的 \(\varepsilon > 0\),存在正整数 \(N\),当 \(n > N\) 时,\(|z_n - z_0| < \varepsilon\),记作 \(\lim_{n \to \infty} z_n = z_0\)。我们称 \(\{z_n\}\) 收敛到 \(\infty\),是指对任给的正数 \(M > 0\),存在正整数 \(N\),当 \(n > N\) 时,\(|z_n| > M\),记为 \(\lim_{n \to \infty} z_n = \infty\)。

\begin{theorem}[与分量的关系]
    \(\lim_{n \to \infty} z_n = z_0\) 的充分必要条件是 \(\{z_n\}\) 的实部和虚部都分别有 \(\lim_{n \to \infty} x_n = x_0\) 和 \(\lim_{n \to \infty} y_n = y_0\)。
\end{theorem}

\begin{definition}[Cauchy列]
    称复数列 \(\{z_n\}\) 称为 Cauchy 列,如果对任给的 \(\varepsilon > 0\),存在正整数 \(N\),当 \(m, n > N\) 时,有 \(|z_n - z_m| < \varepsilon\)。
\end{definition}
\begin{theorem}[与分量的关系]
    设 \(z_n = x_n + \mathrm{i}y_n\),\(z_m = x_m + \mathrm{i}y_m\),那么从等式
\[
|z_n - z_m| = \sqrt{(x_n - x_m)^2 + (y_n - y_m)^2}
\]
知道,\(\{z_n\}\) 是 Cauchy 列的充分必要条件是它的实部 \(\{x_n\}\) 和虚部 \(\{y_n\}\) 都是实的 Cauchy 列,因而从实数域中的 Cauchy 收敛准则立刻得到复数列的 Cauchy 收敛准则:\(\{z_n\}\) 收敛的充要条件是 \(\{z_n\}\) 为 Cauchy 列,由此知道 \(\mathbb{C}\) 是完备的。

\end{theorem}

\section{开集,闭集和紧集}%
\label{sec:open_close_compact}
OMIT.

\section{曲线和域}%
\label{sec:curve_field}

\begin{definition}[连续曲线]
    称$\gamma$为连续曲线,如果 \(\gamma: [a, b] \to \mathbb{C}, \, t \mapsto z = \gamma(t) = x(t) + \mathrm{i} y(t), \quad a \leq t \leq b.\)
    其中\(x(t), y(t)\) 都是 \([a, b]\) 上的连续函数。\\ 
    \textbf{(起点,终点)}设$\gamma: \, [a, b] \to \mathbb{C}$为连续曲线,那么:称 $\gamma\left( a \right) $ 为起点,$\gamma\left( b \right) $为终点。
\end{definition}

\begin{definition}[闭曲线]设\(\gamma: [a, b] \to \mathbb{C} \)为连续曲线,那么:
  称为$\gamma$为闭曲线,如果 \(\gamma(a) = \gamma(b)\)。\\
设\(\gamma: [a, b] \to \mathbb{C} \)为连续曲线,那么:
称$\gamma$为简单曲线/Jordan 曲线,如果 \(\gamma\) 仅当 \(t_1 = t_2\) 时才有 \(\gamma(t_1) = \gamma(t_2)\)。\\
设\(\gamma: [a, b] \to \mathbb{C} \)为连续曲线,那么:
称$\gamma$ 为简单闭曲线/Jordan 闭曲线/围道,如果只有当 \(t_1 = a, t_2 = b\) 时才有 \(\gamma(t_1) = \gamma(t_2)\)。
\end{definition}

\begin{definition}[可求长曲线]
    设 \(z = \gamma(t)\) (\(a \leq t \leq b\)) 是一条曲线。对区间 \([a, b]\) 作分割 \(a = t_0 < t_1 < \dots < t_n = b\),得到以 \(z_k = \gamma(t_k)\) (\(k = 0, 1, \dots, n\)) 为顶点的折线 \(P\),那么 \(P\) 的长度为
\[
|P| = \sum_{k=1}^n |\gamma(t_k) - \gamma(t_{k-1})|.
\]
称曲线 \(\gamma\) 是可求长的,如果不论如何分割区间 \([a, b]\),所得折线的长度都是有界的,且\(\gamma\) 的长度定义为 \(|P|\) 的上确界。
\end{definition}

\begin{definition}[光滑曲线]
    如果 \(\gamma'(t) = x'(t) + \mathrm{i} y'(t)\) 存在,且 \(\gamma'(t) \neq 0\),那么 \(\gamma\) 在每一点都有切线,\(\gamma'(t)\) 就是曲线 \(\gamma\) 在 \(\gamma(t)\) 处的切向量,它与正实轴的夹角为 \(\operatorname{Arg}(\gamma'(t))\)。如果 \(\gamma'(t)\) 是连续函数,那么 \(\gamma\) 的切线随 \(t\) 而连续变动,这时称 \(\gamma\) 为**光滑曲线**。在这种情况下,\(\gamma\) 的长度为
\[
\int_a^b \sqrt{(x'(t))^2 + (y'(t))^2} dt = \int_a^b |\gamma'(t)| dt.
\]

曲线 \(\gamma\) 称为是**逐段光滑**的,如果存在 \(t_0, t_1, \dots, t_n\),使得 \(a = t_0 < t_1 < \dots < t_n = b\),\(\gamma\) 在每个参数区间 \([t_{k-1}, t_k]\) 上是光滑的,在每个分点 \(t_{k-1}, t_k\) 处 \(\gamma\) 的左右导数存在。
\end{definition}

\begin{definition}[连通集]
    平面点集 \(E\) 称为是**连通的**,如果对任意两个不相交的非空集 \(E_1\) 和 \(E_2\),满足
\[
E = E_1 \cup E_2,
\]
那么 \(E_1\) 必含有 \(E_2\) 的极限点,或者 \(E_2\) 必含有 \(E_1\) 的极限点。也就是说,\(E_1 \cap \overline{E_2}\) 和 \(\overline{E_1} \cap E_2\) 至少有一个非空。
\end{definition}

\begin{theorem}[连通性刻画]
     \(\mathbb{C}\) 中的开集 \(E\) 是连通的充分必要条件是 \(E\) 不能表示为两个不相交的非空开集 \(E_1\) 和 \(E_2\) 的并。
\end{theorem}
\begin{proof}
 设开集 \(E\) 是连通的,如果存在不相交的非空开集 \(E_1\) 和 \(E_2\),使得 \(E = E_1 \cup E_2\)。由于 \(E_1\) 中的点都是 \(E_1\) 的内点,\(E_1\) 中的点都是 \(E_2\) 的内点,因此 \(E_1\) 中没有 \(E_2\) 的极限点,\(E_2\) 中也没有 \(E_1\) 的极限点,这与 \(E\) 的连通性相矛盾,这就证明了条件的必要性。反之,如果开集 \(E\) 不是连通的,则必存在不相交的非空集 \(E_1\) 和 \(E_2\),使得 \(E = E_1 \cup E_2\),且 \(E_1\) 中无 \(E_2\) 的极限点,\(E_2\) 中无 \(E_1\) 的极限点。由此可见,\(E_1\) 和 \(E_2\) 均为开集,这就证明了条件的充分性。    
\end{proof}

\begin{theorem}[连通性刻画]
    平面上的非空开集 \(E\) 是连通的充分必要条件是:\(E\) 中任意两点可用位于 \(E\) 中的折线连接起来。   
\end{theorem}

\begin{proof}
先证必要性。设 \(E\) 是平面上一个非空的连通的开集,任取 \(a \in E\),定义 \(E\) 的子集 \(E_1, E_2\) 如下:
\[
E_1 = \{ z \in E: z \text{ 和 } a \text{ 可用位于 } E \text{ 中的折线连接} \},
\]
\[
E_2 = \{ z \in E: z \text{ 和 } a \text{ 不能用位于 } E \text{ 中的折线连接} \}.
\]
显然,\(E = E_1 \cup E_2\),而且 \(E_1 \cap E_2 = \emptyset\)。现在证明 \(E_1\) 和 \(E_2\) 都是开集。任取 \(z_0 \in E_1\),因 \(E\) 是开集,故必有 \(z_0\) 的邻域 \(B(z_0, \delta) \subset E\)。这一邻域中的所有点当然可用一条线段与 \(z_0\) 相连,因而可用位于 \(E\) 中的折线与 \(a\) 相连,即 \(B(z_0, \delta) \subset E_1\),所以 \(E_1\) 是开集。再任取 \(z_0' \in E_2\),则必有 \(z_0'\) 的邻域 \(B(z_0', \delta') \subset E\)。如果此邻域中有点能用一条折线与 \(a\) 点相连,那么 \(z_0'\) 能用线段与该点相连,因而 \(z_0'\) 能用折线与 \(a\) 点相连,这与 \(z_0'\) 的定义矛盾。因而 \(B(z_0', \delta') \subset E_2\),即 \(E_2\) 也是开集。由 \(E\) 的连通性知道,\(E_1, E_2\) 中必有一个是空集。由于 \(a \in E_1\),故 \(E_2 = \emptyset\),这就证明了 \(E\) 中所有点都能用折线与 \(a\) 相连,而 \(E\) 中任意两点可以用经过 \(a\) 的折线相连,证明了必要性。

再证条件的充分性。如果存在两个不相交的非空开集 \(E_1, E_2\),使得 \(E = E_1 \cup E_2\)。任取 \(z_1 \in E_1, z_2 \in E_2\)。由假定,这两点可用 \(E\) 中的折线连接,因而折线中必有一条线段把 \(E_1\) 中的一点与 \(E_2\) 中的一点连接起来。不妨设这条线段连接的就是 \(z_1\) 和 \(z_2\),该线段的参数表示为
\[
z = z_1 + t(z_2 - z_1),
\]
其中 \(t \in [0, 1]\)。今设
\[
T_1 = \{ t \in (0, 1): z_1 + t(z_2 - z_1) \in E_1 \},
\]
\[
T_2 = \{ t \in (0, 1): z_1 + t(z_2 - z_1) \in E_2 \}.
\]
则 \(T_1, T_2\) 是非空的不相交的开集,而且 \(T_1 \cup T_2 = (0, 1)\),这与区间的连通性相矛盾。    
\end{proof}

\begin{definition}[域]
    称非空的连通开集为域。
\end{definition}
从上面的定理知道,域中任意两点必可用位于域中的折线连接起来.
从几何上来看,一个域就是平面上连成一片的开集. 例如,单位圆的内部、上半平面、下半平面等都是域的例子。

\begin{theorem}[Jordan]
一条无界简单闭曲线 \(\gamma\) 把复平面 \(\mathbb{C}\) 分成两个域,其中一个是有界的,称为 \(\gamma\) 的\textbf{内部};另一个是无界的,称为 \(\gamma\) 的\textbf{外部},\(\gamma\) 是这两个域的共同边界。
\end{theorem}

\begin{definition}[单连通区域,多连通区域]
域 \(D\) 称为是单连通的,如果 \(D\) 内任意简单闭曲线的内部仍在 \(D\) 内。不是单连通的域称为多连通的。\\
如果域 \(D\) 是由 \(n\) 条简单闭曲线围成的,就称 \(D\) 是 \(n\) 连通的,简单闭曲线也可以退化成一条简单曲线或一点的。
    
\end{definition}

\section{复变函数的极限和连续性}%
\label{sec:function_lim_contin}

\begin{definition}[单值,多值复变函数]
    设 \(E\) 是复平面上一点集,如果对每一个 \(z \in E\),按照某一规则有确定的复数 \(w\) 与之对应,我们就说在 \(E\) 上确定了一个单值复变函数,记为 \(w = f(z)\);或对 \(z \in E\) 对应的 \(w\) 有几个或无穷多个,则称在 \(E\) 上确定了一个多值函数。
\end{definition}
例如,\(w = z^2\),\(w = e^z\),\(w = z^3 + 1\) 都是确定在整个平面上的单值函数;而 \(w = \sqrt[n]{z}\),\(w = \operatorname{Arg} z\) 则是多值函数。今后若非特别说明,所讲的函数都是指单值函数。

\begin{definition}[映射]
    复变函数是定义在平面点集上的,它的值域也是一个平面点集,因此复变函数也称为映射,它把一个平面点集映成另一个平面点集。与 \(z \in E\) 对应的点 \(w = f(z)\) 称为 \(z\) 在映射 \(f\) 下的像点,反之 \(z\) 就称为 \(w\) 的原像。点集 \(\{f(z): z \in E\}\) 也称为 \(E\) 在映射 \(f\) 下的像,记为 \(f(E)\)。如果 \(f(E) \subset F\),\(F\) 就说 \(f\) 把 \(E\) 映入 \(F\),或者说 \(f\) 是 \(E\) 到 \(F\) 中的映射。如果 \(f(E) = F\),就说 \(f\) 把 \(E\) 映为 \(F\),或者说 \(f\) 是 \(E\) 到 \(F\) 上的映射。
\end{definition}

\begin{definition}[极限]
    对于任给的 \(\varepsilon > 0\),存在与 \(\varepsilon\) 有关的正数 \(\delta\),使得当 \(z \in B(z_0, \delta) \cap E\) 且 \(z \neq z_0\) 时有 \(f(z) \in B(a, \varepsilon)\)。后一种说法也适用于 \(z_0 = \infty\) 的情形。  
\end{definition}
设 \(f\) 是定义在点集 \(E\) 上的一个复变函数,\(z_0\) 是 \(E\) 的一个极限点,\(a\) 是给定的一个复数。如果对任意的 \(\varepsilon > 0\),存在与 \(\varepsilon\) 有关的 \(\delta > 0\),使得当 \(z \in E\) 且 \(0 < |z - z_0| < \delta\) 时有 \(|f(z) - a| < \varepsilon\),就说当 \(z \to z_0\) 时 \(f(z)\) 有极限 \(a\),记作 \(\lim_{z \to z_0} f(z) = a\)。

\begin{theorem}[与分量的关系]
    \(\lim_{z \to z_0} f(z) = a\) 的充分必要条件为
\[
\lim_{\substack{x \to x_0 \\ y \to y_0}} u(x, y) = \alpha, \quad \lim_{\substack{x \to x_0 \\ y \to y_0}} v(x, y) = \beta.
\]  
\end{theorem}
\begin{proof}
    设 \(a = \alpha + \mathrm{i}\beta\),\(z_0 = x_0 + \mathrm{i}y_0\),\(f(z) = u(x, y) + \mathrm{i}v(x, y)\),由下面的不等式
\[
|u(x, y) - \alpha| \leq |f(z) - \alpha| + |v(x, y) - \beta|,
\]
\[
|v(x, y) - \beta| \leq |f(z) - \alpha| + |u(x, y) - \beta|
\]
\end{proof}

\begin{definition}[连续]
    我们说 \(f\) 在点 \(z_0 \in E\) 连续,如果
\[
\lim_{z \to z_0} f(z) = f(z_0).
\]
如果 \(f\) 在集 \(E\) 中每点都连续,就说 \(f\) 在集 \(E\) 上连续。
\end{definition}
\begin{theorem}[与分量的关系]
\(f(z) = u(x, y) + \mathrm{i}v(x, y)\) 在 \(z_0 = x_0 + \mathrm{i}y_0\) 处连续的充要条件是 \(u(x, y)\) 和 \(v(x, y)\) 作为二元函数在 \((x_0, y_0)\) 处连续。
\end{theorem}

\begin{prop}[紧集上的连续函数的性质]
   设 \(E\) 是 \(\mathbb{C}\) 中的紧集,\(f: E \to \mathbb{C}\) 在 \(E\) 上连续,那么
\begin{enumerate}
    \item \(f\) 在 \(E\) 上有界;
    \item \(|f|\) 在 \(E\) 上能取得最大值和最小值,即存在 \(a, b \in E\),使得对每个 \(z \in E\),都有
          \[
          |f(z)| \leq |f(a)|, \quad |f(z)| \geq |f(b)|;
          \]
    \item \(f\) 在 \(E\) 上一致连续。
\end{enumerate} 
\end{prop}
