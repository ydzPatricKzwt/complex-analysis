\part{全纯函数的积分表示}
\section{复变函数的积分}%
\label{sec:fubianhanshudejifen}

\begin{definition}[函数沿曲线可积]
    设 \( z = \gamma(t) \) (\( a \le t \le b \)) 是一条可求长曲线,\( f \) 是定义在 \( \gamma \) 上的函数,沿 \( \gamma \) 的正方向取分点 \( \gamma(a) = z_0, z_1, \dots, z_n = \gamma(b) \),在 \( \gamma \) 中从 \( z_{k-1} \) 到 \( z_k \) 的弧段上任取点 \( \zeta_k \),\( k = 1, \dots, n \)(见图\ref{fig:12-png}),作 Riemann 和
    \begin{equation}
        \label{equ:limanhe}
\sum_{k=1}^n f(\zeta_k)(z_k - z_{k-1}). 
    \end{equation}
    用 \( s_k \) 记弧段 \( \overarc{z_{k-1}z_k} \) 的长度,那么:
称 \( f \) 沿 \( \gamma \) 可积,如果 当 \( \lambda = \max\{s_k : 1 \le k \le n\} \to 0 \) 时,不论 \( \zeta_k \) 的取法如何,和式\ref{equ:limanhe}总有一确定的极限。\\
\textbf{(函数沿曲线的积分).}设\( f \) 沿 \( \gamma \) 可积,那么:称\( \int_\gamma f(z)\mathrm{d}z \)为\( f \) 沿 \( \gamma \) 的积分,如果\[
\int_\gamma f(z)\mathrm{d}z = \lim_{\lambda \to 0} \sum_{k=0}^n f(\zeta_k)(z_k - z_{k-1}).
\]
 \end{definition}

\begin{marginfigure}
    \centering
    \includegraphics[width=1.2\textwidth]{figures/12.png}
    \caption{}
    \label{fig:12-png}
\end{marginfigure}

\begin{prop}[可积的条件]
    设$f=u+\mathrm{i}v$在可求长曲线$\gamma$ 上连续,那么:$f$ 沿$\gamma$ 可积,且
    \[
\int_\gamma f(z)\,\mathrm{d}z = \int_\gamma u\,\mathrm{d}x - v\,\mathrm{d}y + \mathrm{i}\int_\gamma v\,\mathrm{d}x + u\,\mathrm{d}y.
\]
\end{prop}

\begin{prop}[可积的条件]
    如果 \( z = \gamma(t) \) (\( a \le t \le b \)) 是光滑曲线,\( f \) 在 \( \gamma \) 上连续,那么
\[
\int_\gamma f(z)\,\mathrm{d}z = \int_a^b f(\gamma(t))\gamma'(t)\,\mathrm{d}t.
\]
\end{prop}

\begin{prop}[可积的性质]
    如果 \(f, g\) 在可求长曲线 \(\gamma\) 上连续,那么:
\begin{enumerate}
    \item \(\int_{\gamma^-} f(z)\,\mathrm{d}z = -\int_\gamma f(z)\,\mathrm{d}z\),这里,\(\gamma^-\) 是指与 \(\gamma\) 方向相反的曲线;
    \item \(\int_\gamma \bigl(\alpha f(z) + \beta g(z)\bigr)\,\mathrm{d}z = \alpha \int_\gamma f(z)\,\mathrm{d}z + \beta \int_\gamma g(z)\,\mathrm{d}z\),这里,\(\alpha, \beta\) 是两个复常数;
    \item \(\int_\gamma f(z)\,\mathrm{d}z = \int_{\gamma_1} f(z)\,\mathrm{d}z + \int_{\gamma_2} f(z)\,\mathrm{d}z\),这里,\(\gamma\) 是由 \(\gamma_1\) 和 \(\gamma_2\) 组成的曲线。
\end{enumerate}
\end{prop}
\begin{prop}[积分估计]
    如果 \(\gamma\) 的长度为 \(L\),\(M = \sup_{z \in \gamma} |f(z)|\),那么
\[
\left| \int_\gamma f(z)\,\mathrm{d}z \right| \le ML.
\]
\end{prop}

\section{Cauchy积分定理}%
\label{sec:Cauchy_jifendingli}
\begin{theorem}[Cauchy积分定理]
    设 \( D \) 是 \( \mathbb{C} \) 中的单连通域,\( f \in H(D) \),且 \( f' \) 在 \( D \) 中连续,则对 \( D \) 中任意的可求长闭曲线 \( \gamma \),均有
\[
\int_\gamma f(z)\,\mathrm{d}z = 0.
\]
\end{theorem}
\begin{proof}
由 \(\gamma\) 围成的域记为 \(G\),因为 \(f'\) 连续,即 \(\frac{\partial u}{\partial x}, \frac{\partial v}{\partial x}, \frac{\partial u}{\partial y}, \frac{\partial v}{\partial y}\) 连续,故可用 Green 公式。又因 \(f\) 在 \(D\) 中全纯,故 Cauchy-Riemann 方程成立。于是
\begin{align*}
\int_\gamma u\,\mathrm{d}x - v\,\mathrm{d}y &= \iint_G \left( -\frac{\partial v}{\partial x} - \frac{\partial u}{\partial y} \right)\mathrm{d}x\mathrm{d}y = 0, \\
\int_\gamma v\,\mathrm{d}x + u\,\mathrm{d}y &= \iint_G \left( \frac{\partial u}{\partial x} - \frac{\partial v}{\partial y} \right)\mathrm{d}x\mathrm{d}y = 0.
\end{align*}
\end{proof}

\begin{theorem}[Cauchy-Goursat]
    设 \( D \) 是 \( \mathbb{C} \) 中的单连通域,如果 \( f \in H(D) \),那么对 \( D \) 中任意的可求长闭曲线 \( \gamma \),均有
\[
\int_\gamma f(z)\,\mathrm{d}z = 0.
\]
\begin{lemma}
    设 \( f \) 是域 \( D \) 中的连续函数,\( \gamma \) 是 \( D \) 内的可求长曲线。对于任给的 \( \varepsilon > 0 \),一定存在一条 \( D \) 中的折线 \( P \),使得
\begin{enumerate}
    \item \( P \) 和 \( \gamma \) 有相同的起点和终点,\( P \) 中其他的顶点都在 \( \gamma \) 上;
    \item \( \left| \int_\gamma f(z)\,\mathrm{d}z - \int_P f(z)\,\mathrm{d}z \right| < \varepsilon \)。
\end{enumerate}
\end{lemma}
\end{theorem}

\begin{theorem}[减弱条件]
    设 \( D \) 是可求长简单闭曲线 \( \gamma \) 的内部,若 \( f \in H(D) \cap C(\overline{D}) \),则
\[
\int_\gamma f(z)\,\mathrm{d}z = 0.
\]
\end{theorem}

在下面的意义下,Cauchy 积分定理在多连通域内也成立。设 \(\gamma_0,\gamma_1,\dots,\gamma_n\) 是 \(n+1\) 条可求长的简单闭曲线,如果 \(\gamma_1,\dots,\gamma_n\) 都在 \(\gamma_0\) 的内部,\(\gamma_1,\dots,\gamma_n\) 中的每一条都在其他 \(n-1\) 条的外部,这样的 \(n+1\) 条曲线就围成了一个 \(n+1\) 连通域 \(D\),这个域 \(D\) 的边界 \(\gamma\) 由 \(\gamma_0,\gamma_1,\dots,\gamma_n\) 共 \(n+1\) 条曲线组成。\(\gamma\) 的正方向规定如下:当我们沿着 \(\gamma\) 的正方向运动时,\(D\) 总是在我们的左边。这时,对 \(\gamma_0\) 来说是逆时针方向,而对 \(\gamma_1,\dots,\gamma_n\) 则是顺时针方向。对于这样的域和边界,Cauchy 积分定理也成立。

\begin{theorem} 设 \(\gamma_0,\gamma_1,\dots,\gamma_n\) 是 \(n+1\) 条可求长简单闭曲线,\(\gamma_1,\dots,\gamma_n\) 都在 \(\gamma_0\) 的内部,\(\gamma_1,\dots,\gamma_n\) 中的每一条都在其他 \(n-1\) 条的外部,\(D\) 是由这 \(n+1\) 条曲线围成的域,用 \(\gamma\) 记 \(D\) 的边界。如果 \(f \in H(D) \cap C(\overline{D})\),那么
\begin{equation}
\int_\gamma f(z)\,\mathrm{d}z = 0,
\label{eq:3.2.4}
\end{equation}
这里,积分沿 \(\gamma\) 的正方向进行。式 \eqref{eq:3.2.4} 也可写为
\begin{equation}
\int_{\gamma_0} f(z)\,\mathrm{d}z = \int_{\gamma_1} f(z)\,\mathrm{d}z + \dots + \int_{\gamma_n} f(z)\,\mathrm{d}z,
\label{eq:3.2.5}
\end{equation}
式 \eqref{eq:3.2.5} 右端的积分分别沿 \(\gamma_1,\dots,\gamma_n\) 的逆时针方向进行。
\end{theorem}
